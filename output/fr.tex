
\documentclass[10pt, a4paper, roman, french]{moderncv}
\moderncvstyle{classic}                             
\moderncvcolor{purple}                              
\usepackage[utf8]{inputenc}
\usepackage[light]{CormorantGaramond}
\usepackage[T1]{fontenc}
\usepackage[scale=0.75,a4paper]{geometry}
\usepackage{babel}
\usepackage{geometry}
\geometry{hmargin=2.5cm,vmargin=1.5cm}

%----------------------------------------------------------------------------------
%            informations personnelles
%type t' = {
%  title : string;
%  description : string;
%  company : string;
%  location : string;
%  date : string;
%}
%----------------------------------------------------------------------------------
\firstname{Émile}
\familyname{Trotignon}
\mobile{+33 7 82 89 83 58}                          
\extrainfo{Né le 30 juillet 1999}
\email{emile.trotignon@gmail.com}                               
% \photo[64pt][0.4pt]{Image} % 
\begin{document}
	\makecvtitle
	\section{Formation}
	
		\cventry{2019 -- 2020 }{L3 Informatique}{École Normale Supérieure Paris-Saclay}{}{}{}
	
		\cventry{2018 -- 2019 }{L2 Informatique - mathématiques}{Université Lyon 1 Claude-Bernard}{}{}{}
	
		\cventry{2017 -- 2018 }{CPGE MPSI}{Lycée Jean Perrin (Option Informatique au Lycée du Parc)}{}{}{}
	
		\cventry{2016 -- 2017 }{Baccalauréat scientifique}{Lycée La Trinité}{}{}{}
	
	\section{Expérience}
	
		\cventry{Janvier 2020}{ICPC SWERC 2019-2020 }{}{Télécom Paris}{}{Compétition de programmation/algorithmique universitaire. 
                             Participation au sein d'une équipe de trois.
                             Classement de mon équipe : 37 sur 95 équipes répresentant des universités de plusieurs pays européens}
	 
		\cventry{Été 2019}{Développeur stagiaire C\# }{Eternix Ldt.}{Tel Aviv, Israel}{}{Stage de 2 mois. Écriture de shaders HLSL, découverte de DirectX, développement Windows Form, expérience avec OpenCV.
                             Expérience extrêmement enrichissante dans une entreprise étrangère}
	 

	\section{Langues}
		
	    \cvitem{Anglais}{Courant}
		
	    \cvitem{Français}{Maternel}
			
	\section{Compétences techniques}
	
		\subsection{C++ (trés fort)}
			Cours de C++ suivi \`a l'université. Utilisation du C++ pour des projets personnels utilisant des concept plus avancés, ainsi que pour de la programmation compétitive.
	
	                Je sais aussi programmer en C ANSI.
	
		\subsection{Python (intermédiaire)}
			Programmation en Python comme loisir depuis 2013. J'ai également suivi des cours de Python lors de mon année de terminale (spécialité ISN), ainsi que lors de mon année de CPGE .
	
	                       J'ai plusieurs projets en python disponibles sur ma page GitHub :

                         \url{https://github.com/EmileTrotignon?tab=repositories&q=&type=&language=python}
	
		\subsection{C\# (trés fort)}
			Bonne connaissance de C\# : expérience professionelle de ce langage de 2 mois.

                         Deux projets d'apprentissage réalisés dans le cadre de ce stage :

                         \url{https://github.com/EmileTrotignon/InterpolationShaders}

Ce projet affiche une image avec DirectX et permet d'utiliser l'interpolation de Lanczos sur celle-ci.

\url{https://github.com/EmileTrotignon/CSGoban}

Petite application permettant de jouer au jeu de Go.
	
		\subsection{OCaml (fort)}
			Cours de OCaml suivi dans le cadre de l'option informatique en CPGE : ce cours est orienté informatique théorique.
                
                         Lors de mes études à l'ENS, j'ai écris en OCaml un compilateur pour un sous-ensemble du langage C :

                         \url{https://github.com/EmileTrotignon/mcc}
	
		\subsection{Développement web (fort)}
			Front-end : Bonne connaissance de HTML/CSS : j'ai exercé cette compétence professionnellement lors de l'été 2018.
	
	                       Back-end : Experience professionnelle de developpement d'une application Node.js .
	
		\subsection{Unix (intermédiaire)}
			Je sais utiliser un système Unix avec la ligne de commande : manipulation de fichier, Git, SSH 
	
\end{document}